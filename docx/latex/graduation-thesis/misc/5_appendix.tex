%%
% The BIThesis Template for Bachelor Graduation Thesis
%
% 北京理工大学毕业设计(论文)附录 —— 使用 XeLaTeX 编译
%
% Copyright 2020 Spencer Woo
%
% This work may be distributed and/or modified under the
% conditions of the LaTeX Project Public License, either version 1.3
% of this license or (at your option) any later version.
% The latest version of this license is in
%   http://www.latex-project.org/lppl.txt
% and version 1.3 or later is part of all distributions of LaTeX
% version 2005/12/01 or later.
%
% This work has the LPPL maintenance status `maintained'.
%
% The Current Maintainer of this work is Spencer Woo.
%
% Compile with: xelatex -> biber -> xelatex -> xelatex

\unnumchapter{附~~~~录}
\renewcommand{\thechapter}{附录}

% 设置附录编号格式
\ctexset{
  section/number = 附录\Alph{section}
}

% 这里示范一下添加多个附录的方法:

\section{传统区块链私有链建立}
\begin{enumerate}
  \item 建立私链首先需要构建创世块,如下是本课题传统区块链上所使用的创世块genesis.json 
  \begin{center}
    \fbox{
      \parbox{130mm}{
        \{
            "config": \{ \\
            "chainId": 666,
            "homesteadBlock": 0, 
            "eip150Block": 0, \\
            "eip150Hash": "0x0000000000000000000000000000000000000000000000 \par 000000000000000000",\\
            "eip155Block": 0, 
            "eip158Block": 0, 
            "byzantiumBlock": 0, \\
            "constantinopleBlock": 0,
            "petersburgBlock": 0,
            "istanbulBlock": 0,\\
            "ethash": \{\}
        \},
        "nonce": "0x0",
        "timestamp": "0x5ddf8f3e",\\
        "extraData": "0x00000000000000000000000000000000000000000000 \par 00000000000000000000",\\
        "gasLimit": "0x47b760", 
        "difficulty": "0x00002",\\
        "mixHash": "0x000000000000000000000000000000000000000000000000 \par 0000000000000000", \\
        "coinbase": "0x0000000000000000000000000000000000000000",\\
        "alloc": \{ \}, 
        "number": "0x0",
        "gasUsed": "0x0",\\
        "parentHash": "0x00000000000000000000000000000000000000000000000 \par 00000000000000000"
        \}
      }
    }
  \end{center}
  \item 初始化私链
  \begin{center}
    \fbox{
        geth init ./genesis.json --datadir "./chain"
    }
  \end{center} 
  之后所有私链上的相关信息都将保存至chain文件夹中
  \item 启动私链 
  \begin{center}
    \fbox{
        \parbox{130mm}{
            geth --identity "etherum" --rpcaddr 127.0.0.1 --rpc --rpcport "8545" --rpccorsdomain "*"  --maxpeers 2 --rpcapi "personal,eth,net,web3,debug" --networkid 100 --datadir "./chain" --nodiscover --allow-insecure-unlock --dev.period 1 console
        }
    }
\end{center} 
由于GeoHashTile前端使用html编写,通过web3与合约进行交互,为此,需要开启rpc选项,并指定对应端口,才能正确访问合约。
\end{enumerate}
        
   

\section{Truffle合约编译与部署}
\begin{enumerate}
  \item 新建文件夹并初始化;
  \begin{center}
    \fbox{
      \parbox{130mm}{
        mkdir test \\
        truffle init
      }
    }
  \end{center}
  \item 在contracts文件夹中添加合约StoreTraffic.sol;
  \item 在migrations文件夹中添加部署文件; \\
  注意:变量名开头需要大写,且与合约文件名一致。
  \begin{center}
    \fbox{
        \parbox{130mm}{
          const StoreTraffic = artifacts.require("StoreTraffic");\\
          module.exports = function(deployer) \{ \\
            deployer.deploy(StoreTraffic); \\
          \};          
        }
    }
\end{center} 
  \item 更改truffle-config.js文件,监控本地网络;
  \begin{center}
    \fbox{
        \parbox{130mm}{
          module.exports = \{ \\
          networks: \{ \\
            development: \{ \\
              host: "localhost", \\
              port: 8545, \\
              network\_id: "*" // Match any network id \\
            \} 
          \}
        \};
          
        }
    }
\end{center} 
  \item truffle compile 编译合约
  \item truffle migrate 部署合约
\end{enumerate}

\section{树状区块链私有链创世块}
\begin{center}
  \fbox{
    \parbox{130mm}{
      \{ \\
      "config": \{ 
      "chainId": 91036, 
      "homesteadBlock": 0, 
          "eip150Block": 0, 
          "eip155Block": 0, \\
          "eip158Block": 0, 
          "byzantiumBlock": 0,  
          "constantinopleBlock": 0, 
          "petersburgBlock": 0 \\
        \},
        "alloc": \{
        \}, \\
        "coinbase": "0x0000000000000000000000000000000000000000",\\
        "difficulty": "0x20000", \\
        "extraData": "", \\
        "gasLimit": "0xffffffff", \\
        "nonce": "0x0000000000000042", \\
        "mixhash": "0x00000000000000000000000000000000000000000000 \par 00000000000000000000", \\
        "parentHash": "0x0000000000000000000000000000000000000000000 \par 000000000000000000000", \\
        "timestamp": "0x00" \\
      \}
    }
  }
\end{center}