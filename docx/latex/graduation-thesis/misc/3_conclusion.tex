%%
% The BIThesis Template for Bachelor Graduation Thesis
%
% 北京理工大学毕业设计(论文)结论 —— 使用 XeLaTeX 编译
%
% Copyright 2020 Spencer Woo
%
% This work may be distributed and/or modified under the
% conditions of the LaTeX Project Public License, either version 1.3
% of this license or (at your option) any later version.
% The latest version of this license is in
%   http://www.latex-project.org/lppl.txt
% and version 1.3 or later is part of all distributions of LaTeX
% version 2005/12/01 or later.
%
% This work has the LPPL maintenance status `maintained'.
%
% The Current Maintainer of this work is Spencer Woo.
%
% Compile with: xelatex -> biber -> xelatex -> xelatex

\unnumchapter{结~~~~论}
\renewcommand{\thechapter}{结论}

\ctexset{
  section/number = \arabic{section}
}

% 结论部分尽量不使用 \subsection 二级标题,只使用 \section 一级标题
智能汽车的广泛使用,使得车辆之间的联系将会越来越紧密,车辆之间的大量信息,使得车载自组网的应用成为可能。利用车载自组网,将智能车辆组织起来,分享和传递数据。车载自组网需要建立去中心化的ad-hoc网络,而区块链具备去中心化、不可篡改、共识机制等特点,所以将区块链技术用于车载自组网中是一种可行方案。然而,车载自组网需要与地理信息密切结合且节点具备移动性特点,但传统区块链的单链结构并不能够很好的解决上述问题。另外,车载自组网的大量应用都与地理位置信息密切相关,而数据来源于开放的互联网平台,存在被篡改的风险,且传统的地理信息以经纬度的形式在矢量地图表示,在区域信息绑定、信息安全等问题上存在问题。

针对上述问题,本文完成了对区块链多链结构的相关调研,并对树状区块链相关工作进行了介绍,同时在此基础上对当前工作完成了复现。同时,作者对地图信息存储和展现进行了相关研究,并最终实现了基于GeoHash编码的地图信息存储,同时结合区块链的不可篡改特性,保证了地图数据可靠性,并最终通过基于LeafLet的轻量级GeoHashTile前端展现最终的矢量地图,实现了完全基于GeoHash编码的存储和展现,利用GeoHash编码特性,简化了地图存储绑定逻辑,同时减小了请求传递的数量级别,对地图存储展现有所推进。

另外,本文最后一章,对树状区块链进行了相关验证和测试,成功解决合约部署问题,将地图存储和展现的全部工作移植到树状区块链上,同时成功应用了李玮琪的车辆位置验证与信誉评估系统,验证了树状区块链功能的正确性。同时,设计了相关实验和算法,对树状区块链的相关性能进行了测试,在相同的数据存储下,树状区块链会快于传统区块链,同时此结果将随着数据量增多而愈发明显。