%%
% The BIThesis Template for Bachelor Graduation Thesis
%
% 北京理工大学毕业设计(论文)中英文摘要 —— 使用 XeLaTeX 编译
%
% Copyright 2020 Spencer Woo
%
% This work may be distributed and/or modified under the
% conditions of the LaTeX Project Public License, either version 1.3
% of this license or (at your option) any later version.
% The latest version of this license is in
%   http://www.latex-project.org/lppl.txt
% and version 1.3 or later is part of all distributions of LaTeX
% version 2005/12/01 or later.
%
% This work has the LPPL maintenance status maintained'.
%
% The Current Maintainer of this work is Spencer Woo.

% 中英文摘要章节
\zihao{-4}
\vspace*{-11mm}

\begin{center}
  \heiti\zihao{-2}\textbf{\thesisTitle}
\end{center}

\vspace*{2mm}

{\let\clearpage\relax \chapter*{\textmd{摘~~~~要}}}
\addcontentsline{toc}{chapter}{摘~~~~要}
\setcounter{page}{1}

\vspace*{1mm}

\setstretch{1.53}
\setlength{\parskip}{0em}

% 中文摘要正文从这里开始
车载自组网是在交通参与者中构建点对点(ad-hoc)开方式网络,提供去中心化、自组织的数据传输服务。目前,区块链的应用越来越广,车载自组网与区块链结合也逐渐变多,利用区块链去中心化、不可篡改等特性,车载自组网可以实现更加可靠的信息传输与交互,但由于车载自组网的移动性以及与地理位置信息紧密结合等特性,导致传统区块链并不能很好的应用于车载自组网上。另外,车载自组网的大量应用都与地理位置信息密切相关,而数据来源于开放的互联网平台,存在被篡改的风险。针对上述两个问题,本文对区块链结构进行相关调研,并对树状区块链进行介绍和复现,同时完成基于该区块链的地图存储和展现。

首先,本文实现了基于GeoHash编码的地图存储与展现,通过智能合约存储地图数据,前端通过web3请求合约数据,最终通过GeoHashTile对地图数据进行呈现,  其中地图数据完全转换为GeoHash编码的GeoJson格式,完成了对传统地图经纬度数据的替换,充分利用了区块链特性保证地图数据的安全性、可塑性以及网络中的同步性。

其次,本文中对树状区块链的相关工作进行了复现,并在此基础上完成基于GeoHash编码的地图存储和展现的部署,并利用李玮琪的车辆位置验证与信誉评估系统验证了树状区块链的正确性,同时对树状区块链存储效率进行了分析。

\vspace{4ex}\noindent\textbf{\heiti 关键词:车载自组网;区块链结构;地图存储与展现;GeoHash编码}
\newpage

% 英文摘要章节
\vspace*{-2mm}

\begin{spacing}{0.95}
  \centering
  \heiti\zihao{3}\textbf{\thesisTitleEN}
\end{spacing}

\vspace*{5mm}

{\let\clearpage\relax \chapter*{
  \zihao{-3}\textmd{Abstract}\vskip -3bp}}
\addcontentsline{toc}{chapter}{Abstract}
\setcounter{page}{2}

\setstretch{1.53}
\setlength{\parskip}{0em}

% 英文摘要正文从这里开始
The vehicular Ad-hoc Networks(VANETs) is to construct an ad-hoc open mode network in traffic participation to provide decentralized and self-organized data transmission services. Currently, the application of blockchain is becoming more and more widespread, and the combination of VANETs and blockchains has gradually increased. Using the characteristics of blockchain decentralization and uncorrectable changes, VANETs can achieve more reliable information. Transmission and interaction, but due to the mobility of the VANETs and its close integration with location information, the traditional blockchain chain cannot be well embedded in the VANETs. In addition, a large number of applications of the VANETs are related to location information, and the open Internet platform for data changes has the risk of being tampered with. In response to the above two problems, the relevant research on the structure of the blockchain chain is carried out, the tree-like blockchain chain is introduced and reproduced again, and the map storage and display based on the blockchain is completed at the same time.

First of all, this article realizes the storage and display of the map based on GeoHash encoding. The map data is stored through the smart contract. The front end requests the contract data through web3. Finally, the map data is presented through GeoHashTile. The map data is completely converted to the GeoJson format of GeoHash encoding. It replaces the traditional map latitude and longitude data, and makes full use of the characteristics of the blockchain to ensure the security, plasticity and synchronization of the map data.

Secondly, this article reproduces the related work of the tree block chain, and on this basis, completes the deployment of the storage and display of the map based on GeoHash encoding, and uses the vehicle location verification and reputation evaluation system of Li Weiqi to verify the tree. The correctness of the blockchain and the storage efficiency of the tree-shaped blockchain are also analyzed.

\vspace{3ex}\noindent\textbf{Key Words: VANET; Blockchain structure; Map storage and display ;GeoHash}
\newpage
