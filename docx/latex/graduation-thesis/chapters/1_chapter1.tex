%%
% The BIThesis Template for Bachelor Graduation Thesis
%
% 北京理工大学毕业设计(论文)第一章节 —— 使用 XeLaTeX 编译
%
% Copyright 2020 Spencer Woo
%
% This work may be distributed and/or modified under the
% conditions of the LaTeX Project Public License, either version 1.3
% of this license or (at your option) any later version.
% The latest version of this license is in
% http://www.latex-project.org/lppl.txt
% and version 1.3 or later is part of all distributions of LaTeX
% version 2005/12/01 or later.
%
% This work has the LPPL maintenance status maintained'.
%
% The Current Maintainer of this work is Spencer Woo.
%
% 第一章节

\chapter{引言}

\section{课题背景}
% 这里插入一个参考文献,仅作参考
随着智能车的发展与普及,车辆间的交互将变得必不可少,为了营造更加安全高效的车辆交通环境,需要建立车辆间以及车辆与路侧节点的特殊自组网,也就是所谓的“车联网”。在“车联网”中,一大主要应用技术就是车载自组网。车载自组网,是车辆或者路侧节点以自组织的形式构成的多跳网络,为了进行车辆间或车辆与路侧节点间的数据交互\cite{朱雪田20195g}。基于车载自组网,交通参与者之间可以相互通信,形成移动Ad-Hoc网络,并提供去中心化、自组织的数据传输服务\cite{sakiz2017survey}。

区块链最初是作为比特币的基础技术而开发的,它以其创建运行在数百万种设备上的庞大的,全球分布的分类账的能力而迅速成名,该分类账能够记录任何有价值的东西。区块链本质上是一种数字的,分布式的交易分类账,在网络的每台成员计算机上维护着相同的副本。各方都可以查看以前的条目并记录新条目。区块链以轻松、自动化和分散的方式记录的能力引起了各类初创公司和金融服务行业的兴趣,它们正在与多个领域进行融合。\cite{oparah3ways,kar2016estonian}。随着区块链技术的发展与普及,从其活跃的金融市场起,其高度重要性逐渐吸引了不同部门的组织的关注,从移动支付解决方案到医疗保健应用,Blockchain导致了成千上万的新工作岗位和新创业公司的发展\cite{dorri2017blockchain}。

目前,区块链与物联网领域结合的愈发紧密\cite{袁勇2016区块链技术发展现状与展望},区块链上部署相关合约,可以实现无需第三方见证的可信交易,并能够以相同的确定性实现相同的功能,区块链自身具有去中心化、分布式特性,这与车载自组网特性相符,将区块链技术与车载自组网结合起来,能够有效利用区块链的不可篡改性以及可追溯性等特性,解决车载自组网中车辆节点验证问题。

然而,车载自组网中存在很多问题,如车辆节点具备移动特性、车辆网络拓扑结构不断变化,同时,车载自组网中的应用对网络稳定性要求较高,传统的区块链结构已经不能满足上述需求\cite{liu2019kalman}。传统区块链是单链的链式结构,要求所有节点必须存在于同一区块链中,大量的车辆节点将会对区块链存储等问题造成极大影响,而且,传统区块链并没有直接与位置信息联系起来,不满足车辆节点的移动特性。因此,考虑在区块链中增添位置信息,更改传统区块链单链结构,增强区块链对车载自组网的适用性,才能扩展区块链在车载自组网中的应用场景。基于车载自组网的路况查询、事故预警等应用与位置信息联系紧密,攻击者可以伪造地图信息欺骗达到利己的目的,因此,考虑利用区块链特性,将地图信息存储至区块链上,方便多车辆协作,防止数据篡改,为位置信息等提供安全性保障\cite{yang2019maritime}。

\section{相关调研}
\subsection{区块链与智能合约}
区块链是一个共享的不可更改的总账,它用于记录交易、跟踪资产以及建立信任\cite{王震2019面向大数据应用的区块链解决方案综述}。这些资产可以是有形的(房屋,汽车,现金,土地等)或者无形的(知识产权,专利,版权,品牌等)。几乎所有有价值的东西都可以在区块链网络上进行跟踪和交易,从而降低了风险并削减了所有相关成本。交易业务依靠信息,收到的越快、越准确,就会越好。区块链是传递该信息的理想选择,因为它可以提供在不可更改的总账中的即时、共享且完全透明的信息,这类信息同时具备一定的安全性,只能由获得许可的网络成员访问。另外,区块链网络也可以跟踪订单、付款、账户、生产等等。而且因为成员之间可以共享信息,所以成员可以查看到交易端到端的所有细节,从而带来更好的效率。

区块链有三个关键要素,分布式分类账技术、不可篡改以及智能合约。分布式分类账技术是指所有网络参与者都可以访问分布式分类账以及不可变的交易记录,且所有交易仅记录一次,从而消除传统业务网络中大量的重复工作和数据。不可篡改是指,区块链将交易记录到共享的分类账上后无法被篡改,如果交易中包含错误,则必须添加新交易来撤销该错误。

智能合约是存储在区块链上的程序,可以在满足预定条件的情况下运行,它们通常是自动执行的脚本,以便所有参与者无需任何中介机构的参与就可以立即结果,极大保证安全性。智能合约代码语句十分简单,当预定条件已经得到满足并完成验证时,区块链网络将执行对应动作,比如释放资金,发出凭证等,然后交易完成时更新区块链。由智能合约完成的交易也是无法更改的,只有获得许可的参与者才能看到结果。本文的课题是基于以太坊平台来部署智能合约,从而进行相关研究。

\subsection{地理数据表示}
随着移动用户的剧增,交互式信息地图服务需求正在逐渐增多,矢量数据地图正在兴起。在矢量地图数据中,矢量数据可以在所有放缩水平下以不同的颜色正确显示和区分特征,使地图展现更加丰富\cite{zhou2016virtual}。目前矢量地图主要利用Web Map Tile Service(WMTS)框架\cite{garcia2013ols},它可以解决矢量数据分布不均和传输延迟长的问题,在此框架中,地图矢量数据需要被划分为图块,然后根据客户端的请求返回对应图块数据给客户端,通过它们各自的坐标在客户端重新组合,所以矢量数据的编码方式是影响其传输性能和可用性的关键因素。

XML和JSON是web应用程序中常用的两种矢量数据编码方法\cite{wang2011improving}。XML(可扩展标记语言)是一种标记语言,用作Internet信息交换的标准\cite{braysperberg}。它具有良好的语义和可扩展性,并且可以灵活地表示数据,但由于XML使用重量级语法,因此会导致其格式复杂且大小较大,不利于上述地图数据传输。JSON是一种易于读写的轻量级数据表示格式,GeoJson是其中一种轻量级的数据表示方法,用于编码各类地理数据结构,可用于简单表示地理信息\cite{li2015performance}。

传统的矢量地图如Google Maps\cite{kahle2013ggmap},采用二维数据经纬度来表示地理信息,这是一种球面坐标系统。Geohash是一种新型的地址编码方式,它使用Base32编码成一维的字符串代替二维的经纬度数据。与Google Maps经纬度编码相比,GeoHash将二维空间查询转换为一维字符串匹配,利用此优势,GeoHash编码可以实现时间复杂度为O(1)快速查询\cite{liu2014geohash}。与Bing Maps\cite{teslya2014web}相比,GeoHash编码使用Base32编码方法,同一前缀有32个不同子序列,而使用Base4的Bing Maps同一前缀下只有4个子序列。当前,如Google Maps等主流地图产品采用经纬度表示地理数据坐标位置,而索引则采用另一种方法,GeoHash编码可以将二者结合起来,根据编码长度的不同,GeoHash可以同时表示图块的索引范围和坐标,从而可以实现统一编码。同时,GeoHash在编码保留了相对地理位置信息,位数较长的GeoHash编码一定位于对应前缀GeoHash编码所对应的区域中,可以很好的解决区域与具体信息绑定问题。

本文将GeoHash编码用于地图存储与展现上,完全替换传统地图的经纬度数据表示形式,方便了区域信息绑定和查询。

\subsection{区块链结构}
传统区块链技术基于区块,整个网络同时只有一条单链,基于PoW共识机制出块无法并发执行,无法满足车载自组网对高并发操作和网络稳定性的需求,传统区块链也不具备移动性,无法与地理位置信息进行绑定,不能有效利用车载网中地理信息的区域化特性。另外,传统区块链的单链结构,要求所有节点必须在同一区块链中,这将会导致节点数目和数据量过大,不满足车载自组网中车辆节点的移动特性,且一旦出现网络分区,就会对整个区块链产生很大影响。目前采用分片技术更改原始单链的链式结构是解决上述问题的主流方法\cite{global2020}。当前也有多链结构的相关工作。

根据多链结构是否改变底层区块链结构可分为应用多链和结构多链。应用多链,即在应用层面建立不同功能的多个区块链,没有修改底层区块链结构。结构多链,即根据需求对区块链底层结构进行调整的多链结构。
\subsubsection{应用多链}
Shrestha等人\cite{shrestha2019regional}研究了区域区块链在车载自组网中的应用,它是一个由物理边界区域内的节点共享的区块链。如果将区域区块链用于地理有限的区域,由于较小的传播延迟,可以极大减少端到端的消息延迟,提高区块链中节点交易效率,从理论上说,可以将整个区块链分为多个区域区块链,但该文仅从一个区域区块链内部进行了相关研究,并没有设计跨区域和跨链交易的相关内容。

Hirtan等人\cite{hirtan2019blockchain}提出了一种给予隐私保护的医疗保健系统,包含专用链和公用链两种区块链模型。专用链,即侧链,用于保存有关患者的真实身份信息;公用链,即主链,用于存储有关患者健康的信息以及标有临时ID的数据,从而实现了隐私数据保护和可用数据公开访问。通过特定节点掌握患者临时ID和真实身份的关联,来实现两个区块链数据的传递。

Ocha等人\cite{sestrem2020cost}提出了一种使用侧链使智能电网具有可扩展性和适应性的区块链架构,将开放式智能电网协议(OSGP)集成三个区块链中,分别为BlockPRI、BlockSEC和BlockTST。其中,BlockPRI是用于存储每个用户的隐私,BlockSEC存储用户的数据,BlockTST管理和验证有关消费者-消费者和消费者-公司之间的能源贸易的信息。每个区块链对应一个功能,从整体上保证智能电网网络的隐私性和安全性。

\subsubsection{结构多链}
结构多链是根据需求对区块链底层结构进行调整的多链结构,根据其组成特点,结构多链又可以分为并发多链和层次多链。并发多链结构扁平,各个子链可以根据需求负责相同或者不同的功能,相互之间地位平等。而层次多链属于多层级结构,至少具备上下层级关系的子链结构,各个层级负责的功能不同,地位也不同。
\paragraph{并发多链}
Youngjune等人\cite{parkmonoxide}提出异步共识区域,该区域在不影响分散性或安全性的前提下线性扩展了区块链系统。通过建立多个独立并行的单链共识区域来构成并发多链的区块链结构。各个共识组地位平等,大部分交易只在组内完成。分组按指定规则命名,跨组交易采用异步方式将中继事务发送到目标区域,而不是整个网络,减轻了网络负载。此外,为了确保每个区域中的有效采矿能力与整个网络处于同一水平,采用了Chu-ko-nu的改进PoW挖矿方式,这也同时保证了分组抵御攻击的能力。但其共识组分区方法没有考虑实际地理位置信息,不适用于车载自组网。

Zamani等人 \cite{zamani2018rapidchain} 提出了基于分片的公共区块链协议,它将节点划分为多个较小的称为委员会的节点组,节点组在不相交的交易块上并行操作,并维护不相交的独立账本,也就是分片,分片由每个成员以区块链的形式存储。为了解决节点频繁移入移出对网络造成的映像,将委员会将分为活跃和不活跃两个分类,节点创建后,第一次进入委员会需要加入活跃类,再次进入或转移时需要加入不活跃类。RapidChain对分片和跨片交易处理,委员会重组,节点移动都有较好的处理,但也存在一些不足。委员会内节点数量固定,缺少灵活性;委员会构建和重构时没有考虑地理因素,增加了更新时间,不适用于车载自组网。

Feng X等人 \cite{feng2019pruneable} 改进的Rollerchain中利用分片技术和PBFT(拜占庭式容错)算法提出了一种基于可分片的区块链协议(PSRB)。PSRB协议采用分片技术,使得PoW协商机制能够通过特定的社区分配规则和网络的分片功能实现事务的并发管理。社区分配规则为每个节点计算PoW难题,然后根据其nonce分配给某个社区。与现有的区块链协议不同,PSRB协议在每个节点和主链只存储区块头链和部分区块时,保证了系统的安全性,避免了协议的体积膨胀和容量膨胀问题。由于nonce计算的随机性,会出现社区内节点数目差别过大的问题,该方法没有说明对节点处理的方法,且没有结合地理因素,也不适用于车载自组网。

\paragraph{层次多链}
Byung等人 \cite{jo2018hybrid} 将物联网与基于区块链的智能合约相结合,用于结构健康监视(SHM),定义新颖,高效,可扩展和安全的分布式网络,增强运营安全性。该在这个区块链物联网网络中,通过将本地集中和全球分散的分布分为核心和边缘网络,激活了本地集中和全球分散分布的特征,并提供了本地数据库的原始数据存储和核心区块链网络的结构事件存储两种类型的存储。边缘节点充当查询实时响应的集中式服务器,并提供低延迟和带宽使用率,核心网络由具有高存储容量的miner节点组成,负责生成新块、验证工作证明(PoW),并包含用于自主决策的智能合约,这种划分提高了系统的效率和可伸缩性且可以有效地模拟结构的自主监测和控制,但核心网络十分庞大是主要问题 。核心网络的庞大影响移动节点的交易效率,且该区块链也没有考虑地理因素,不适用于车载自组网。

Liu等人 \cite{liu2019mathsf} 为解决部署区块链和处理事务负担过大,提出了LightChain的轻量级区块链系统,该系统由API层,LightChain层, 缓存层和存储层等4层框架构成。API层用来读写本地数据和构造区块链交易;LightChain层通过数字签名验证和附件验证用来验证交易的有效性和完整性;缓存层是为了加速对调用的响应而设计的,它包含本地操作、未验证的块和有用的块;存储层通常由资源丰富的设备提供服务,为上层提供持久存储服务。该结构,解决了部署区块链和处理事务的过重负担,但由于其并没有考虑地理因素,不能很好的应用于车载自组网,但其缓存机制,可以参考,用于加速对调用的响应。

Pajooh等人\cite{honar2021multi}提出了一种多层区块链安全模型来保护物联网网络,同时利用群集的概念来简化多层架构,通过使用模拟退火和遗传算法相结合的混合进化算法来定义物联网K-未知簇,选择的群集头负责本地身份验证和授权,增强了网络认证机制的安全性,显示出更适合的平衡网络延迟和吞吐量。

Chao Q等人\cite{qu2018blockchain}提出了一个具有分层、交叉和自组织区块链结构(BCS)的框架,以建立物联网和区块链之间的关系来验证设备可信度。该工作主要包括三个部分:第一部分,上层区块链节点管理一个下层的区块链节点,不同的区块链网络构成一个层次关系;第二部分,底层的寄存器数据依次传输到上层区块链节点,并途经区块链同样记录数据,第三部分,沿着源设备到目标设备形成验证链。

Mbarek等人\cite{mbarek2019mbs}提出了一个多级区块链框架,以增强物联网应用中的隐私和数据安全。多层次模型侧重于提高响应时间和资源利用率。方案中定义了移动代理来执行哈希函数、实现加密、部署聚合和解密。移动代理在区块链和物联网之间传输,以完成所需的任务。该多层次框架由微观层次(物联网设备层次)、中观层次区块链(物联网的簇头层次)和宏观层次区块链(平台层次或物联网的最高层次)等三层次组成。

Oktian等人\cite{oktian2020hierarchical}提出了一个可扩展的物联网双层分层区块链体系结构。第一层是基于实用拜占庭容错(PBFT)共识来处理高吞吐量的核心引擎,第二层分别由支付引擎、计算引擎和存储引擎组成,用来管理底层的从属引擎(子引擎)。该作者将这些子引擎的多个实例部署到尽可能多的位置,并尽量靠近物联网设备所在的物联网域,以应对大量节点。此外,为了进一步扩展所提出的体系结构的可伸缩性,该作者还在核心引擎上提供了额外的可伸缩性特性,如请求聚合、请求优先级以及子引擎并行性。

\paragraph{智能合约多链}
Ochôa等人\cite{sestrem2020cost}提出了侧链结构,由BlockPRI、BlockSEC、BlockTST三个不同的区块链,使用了三个区块链来确保系统的隐私性,安全性和信任性。BlockPRI存储每个用户的隐私首选项。 BlockSEC存储用户的数据。最后,BlockTST管理并验证有关消费者-生产者与消费者-公司之间的能源贸易的信息。为了保证区块链之间的通信,需要通过智能合约维护多链数据一致性。

\subsection{论文的研究内容及贡献}
本文将传统地图的经纬度数据完成转变为GeoHash编码,并将存储至区块链上,并通过GeoHashTile前端将地图数据进行呈现,利用区块链不可篡改特性,极大的保护了车载自组网中地图数据安全。另外,由于全局同步与共识速度的限制,传统区块链发挥车载自组网的高动态性与地理位置相关等特性。为了解决上述问题,本文还对区块链多链结构进行了相关调研,同时介绍了与地理信息相关的树状区块链并对其进行了复现,同时将地图存储与展现应用于树状区块链上,验证了树状区块链的可行性与正确性,最后对树状区块链进行了一些测试工作。

(1) 准备阶段,通过复现前人相关工作,熟悉了区块链的相关知识。在目前最新的开发和测试环境下,对其工作进行复现,途中遇到很多问题,编写了相关文档,为后人提供帮助;

(2)  成功将地图数据以GeoHash编码格式存储至传统区块链上,并实现区域信息绑定,快速查询指定区域地图数据;

(3) 成功将存储的地图数据以GeoHashTile前端展现,并成功运行车辆位置验证及信誉评估系统,完成了对地图数据正确性的验证;

(4) 将上述工作全部移植到树状区块链上,成功验证树状区块链的正确性;

(5) 对树状区块链进行了相关测试工作。